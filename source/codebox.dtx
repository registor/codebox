% \iffalse meta-comment
% !TEX program  = XeLaTeX
%<*internal>
\iffalse
%</internal>
%<*readme>
Create code box and viewer based on tcolorbox and minted/listings
=======

`codebox` is a LaTeX3 package developed based on tcolorbox and minted/listings, which provides environments `codebox` and `codeview` to typset with environment body, and macros `\codefile` and `\cvfile` to typeset programming source code from a file in a fancy box.

The starred version of all environments and macros are provided at the same time, such as `codebox*`, `codeview*`, `\codefile*` and `\cvfile*` which adds comment at bottom of the fancy box.

You can read the manual (in Chinese/English) for more details and examples.

Contributing
------------

1. github repository:
    1. repository: [codebox](https://github.com/registor/codebox)
    2. Issues and pull requests are welcome. [issue](https://github.com/registor/codebox/issues) or [pull request](https://github.com/registor/codebox/pulls).

2. gitee repository:
    1. repository: [codebox](https://gitee.com/nwafu_nan/codebox)
    2. Issues and pull requests are welcome. [issue](https://gitee.com/nwafu_nan/codebox/issues) or [pull request](https://gitee.com/nwafu_nan/codebox/pulls).

基于tcolorbox的终端模拟宏包
=======

`codebox`是一个基于tcolorbox的用LaTeX3开发的终端模拟LaTeX宏包,它提供了用于排版程序源代码盒子的环境和命令。

其中`codebox`和`codeview`环境用于根据环境内容排版代码,`\codefile`和`\cvfile`命令用于根据文件内容排版代码。

`codebox`同时提供了各环境和命令的星号版本,如`codebox*`和`codeview*`,`\codefile*`和`\cvfile*`,用于为代码盒子添加底线注释。

可以通过阅读宏包手册(中文/英文)以也解该宏包更多的使用细节和使用样例。

###  参与贡献
---------------------

1. github仓库:
    (1). 仓库地址: [codebox](https://github.com/registor/codebox)
    (2). Issues和PR: [issue](https://github.com/registor/codebox/issues) or [pull request](https://github.com/registor/codebox/pulls).

2. gitee仓库:
    (1). 仓库地址: [codebox](https://gitee.com/nwafu_nan/codebox)
    (2). Issues and PR: [issue](https://gitee.com/nwafu_nan/codebox/issues) or [pull request](https://gitee.com/nwafu_nan/codebox/pulls).

Copyright and Licence
---------------------

    Copyright (C) 2020-2022 by Nan Geng <nangeng@nwafu.edu.cn>
    ----------------------------------------------------------------------

    This work may be distributed and/or modified under the
    conditions of the LaTeX Project Public License, either
    version 1.3c of this license or (at your option) any later
    version. This version of this license is in
       http://www.latex-project.org/lppl/lppl-1-3c.txt
    and the latest version of this license is in
       http://www.latex-project.org/lppl.txt
    and version 1.3 or later is part of all distributions of
    LaTeX version 2005/12/01 or later.

    This work has the LPPL maintenance status "maintained".

    The Current Maintainer of this work is Nan Geng.

    This package consists of the file  codebox.dtx,
                 and the derived files codebox.sty,
                                       codebox.pdf,
                                       codebox.ins,
                                       README.md (this file).

%</readme>
%<*internal>
\fi
\begingroup
  \def\temp{LaTeX2e}
\expandafter\endgroup\ifx\temp\fmtname\else
\csname fi\endcsname
%</internal>
%<*install>

\input ctxdocstrip %

\let\MetaPrefix\relax

\preamble

    Copyright (C) 2020-2022 by Nan Geng <nangeng@nwafu.edu.cn>
--------------------------------------------------------------------------

    This work may be distributed and/or modified under the
    conditions of the LaTeX Project Public License, either
    version 1.3c of this license or (at your option) any later
    version. This version of this license is in
       http://www.latex-project.org/lppl/lppl-1-3c.txt
    and the latest version of this license is in
       http://www.latex-project.org/lppl.txt
    and version 1.3 or later is part of all distributions of
    LaTeX version 2005/12/01 or later.

    This work has the LPPL maintenance status "maintained".

    The Current Maintainer of this work is Nan Geng.

--------------------------------------------------------------------------

\endpreamble

\postamble

    This package consists of the file  codebox.dtx,
                 and the derived files codebox.sty,
                                       codebox.pdf,
                                       codebox.ins,
                                       README.md.
\endpostamble

\declarepostamble\emptypostamble
\endpostamble

\def\MetaPrefix{-- }


\let\MetaPrefix\DoubleperCent

\generate
  {
%</install>
%<*internal>
    \usedir{source/xelatex/codebox}
    \file{codebox.ins} {\from{\jobname.dtx}{install}}
%</internal>
%<*install>
    \usedir{xetex/xelatex/codebox}
    \file{codebox.sty} {\from{\jobname.dtx}{package}}
    \nopreamble\nopostamble
    \usedir{doc/xelatex/codebox}
    \file{README.md}   {\from{\jobname.dtx}{readme}}
  }

\endbatchfile
%</install>
%<*internal>
\fi
%</internal>
%<package>\NeedsTeXFormat{LaTeX2e}[2020/10/01]
%<package>\RequirePackage{expl3}
%<+package>\GetIdInfo$Id: codebox.dtx 1.0.4 2022-01-28 08:00:00 +0800 Nan Geng <nangeng@nwafu.edu.cn> $
%<package>  {Code Box with tcolorbox and minted/listings.}
%<package>\ProvidesExplPackage{\ExplFileName}
%<package>  {\ExplFileDate}{\ExplFileVersion}{\ExplFileDescription}
%<*driver>
\documentclass{ctxdoc}
\usepackage{minted}
\usepackage{listings}
\usepackage{xcolor}
\usepackage{xcolor-material}
\usepackage{codebox}
\renewcommand*\marg[1]{\{\meta{#1}\}}
\renewcommand*\oarg[1]{[\meta{#1}]}
\renewcommand*\parg[1]{(\meta{#1})}
\NewDocumentCommand{\init}{+v}{\hspace{\fill}初始值~=~\textcolor{blue}{\bfseries#1}}
\DeclareDocumentCommand\opt{m}{\texttt{#1}}
\DeclareDocumentCommand\kvopt{mm}
  {\texttt{#1\breakablethinspace=\breakablethinspace#2}}
\def\breakablethinspace{\hskip 0.16667em\relax}
\begin{document}
%<!--CODEDOC-->  \DisableImplementation
  \EnableImplementation
  \DocInput{\jobname.dtx}
  \IndexLayout
  \PrintChanges
  \PrintIndex
\end{document}
%</driver>
% \fi
%
% \changes{v1.0.0}{2021/12/25}{开始编写模板}
%
% \CheckSum{313}
% \GetFileId{codebox.sty}
%
% \CharacterTable
%  {Upper-case    \A\B\C\D\E\F\G\H\I\J\K\L\M\N\O\P\Q\R\S\T\U\V\W\X\Y\Z
%   Lower-case    \a\b\c\d\e\f\g\h\i\j\k\l\m\n\o\p\q\r\s\t\u\v\w\x\y\z
%   Digits        \0\1\2\3\4\5\6\7\8\9
%   Exclamation   \!     Double quote  \"     Hash (number) \#
%   Dollar        \$     Percent       \%     Ampersand     \&
%   Acute accent  \'     Left paren    \(     Right paren   \)
%   Asterisk      \*     Plus          \+     Comma         \,
%   Minus         \-     Point         \.     Solidus       \/
%   Colon         \:     Semicolon     \;     Less than     \<
%   Equals        \=     Greater than  \>     Question mark \?
%   Commercial at \@     Left bracket  \[     Backslash     \\
%   Right bracket \]     Circumflex    \^     Underscore    \_
%   Grave accent  \`     Left brace    \{     Vertical bar  \|
%   Right brace   \}     Tilde         \~}
%
%
% \title{\bfseries\pkg{codebox}:代码盒子排版宏包\\ \bfseries\pkg{codebox}:progrmming source code box}
% \author{耿楠\\ \path{nangeng@nwafu.edu.cn}}
% \date{\filedate\qquad\fileversion\thanks{\url{https://github.com/registor/codebox}}\thanks{\url{https://gitee.com/nwafu_nan/codebox}}}
% \maketitle
%
% \begin{documentation}
%
% \begin{abstract}
%
% \pkg{codebox} 是一个基于\pkg{tcolorbox}用\LaTeX3 开发的 \LaTeX 宏包,它提供用于
% 排版源代码盒子的环境和命令。
% 其中,\env{codebox}和\env{codeview}环境用于根据环境内容排版代码,
% \tn{codefile}和\tn{cvfile}命令用于根据文件内容排版代码。
% 同时,\pkg{codebox}还提供了各环境和命令的星号版本,
% 如\env{codebox*}和\env{codeview*},\tn{codefile*}和\tn{cvfile*},
% 用于为代码盒子添加底线注释。
%
% \end{abstract}
%
% \tableofcontents
%
% \section{简介}
%
% \pkg{codebox} 是一个基于\pkg{tcolorbox}用\LaTeX3 开发的 \LaTeX 宏包,
% 它可以根据用户设置使用\pkg{minted}或\pkg{listings}宏包实现代码盒子排版。
%
% 该宏包提供了\env{codebox}和\env{codeview}两个环境和
% \tn{codefile}和\tn{cvfile}两个命令以排版代码盒子。
%
% 同时,\pkg{codebox}宏包还提供环境和命令的星号版本,
% 用于为代码盒子添加底线注释。
%
% \section{用户接口}
%
% \subsection{\env{codebox}和\env{codebox*}环境}
%
% \begin{function}[added=2021-12-25,updated=2021-12-25]{codebox,codebox*}
%   \begin{syntax}
%     \tn{begin}\{codebox\}\oarg{外观选项}\Arg{盒子标题}
%     .....
%     \tn{end}\{codebox\}
%     \tn{begin}\{codebox*\}\oarg{外观选项}\Arg{盒子标题}
%     .....
%     \tn{end}\{codebox*\}
%   \end{syntax}
%   使用环境内容排版代码盒子。
%   可以通过\Arg{盒子标题}为代码盒子设置标题。
%
%   \oarg{外观选项}中通过key-value方式设置代码盒子外观。
%
%   星号环境\texttt{codebox*}用于为代码盒子添加底线注释,
%   此时,需要在\oarg{外观选项}中用
%   \kvopt{\meta{comments}}{\meta{底线注释}}选项提供\textbf{底线注释}内容。
%
%   代码盒子的外观也可以通过\tn{codeset}命令的逗号分隔
%   key-value列表进行设置。
%
% \end{function}
%   \begin{Verbatim}[frame=none,numbers=left,gobble=5]
%     \centering
%     \begin{codebox}{代码标题}
%       #include <stdio.h>
%       #include <stdlib.h>
%
%       int main(void)
%       {
%           printf("Hello World!\n");
%
%           return 0;
%       }
%     \end{codebox}
%   \end{Verbatim}
%
%   \begin{center}
%     \begin{minipage}{0.85\textwidth}
%       \codefile{代码盒子标题}{test.c}
%     \end{minipage}
%   \end{center}
%
% \subsection{\tn{codefile}和\tn{codefile*}命令}
%
% \begin{function}[added=2021-12-25,updated=2021-12-25]{\codefile,\codefile*}
%   \begin{syntax}
%     \tn{codefile}  \oarg{外观选项} \Arg{盒子标题} \Arg{代码文件}
%     \tn{codefile*} \oarg{外观选项} \Arg{盒子标题} \Arg{代码文件}
%   \end{syntax}
%
%   用于排版代码盒子,源代码来自于\Arg{代码文件},
%   可以通过\Arg{盒子标题}为代码盒子设置标题。
%
%   \oarg{外观选项}中通过key-value方式设置代码盒子外观。
%
%   星号命令\texttt{codefile*}用于为代码盒子添加底线注释,
%   此时,需要在\oarg{外观选项}中用
%   \kvopt{\meta{comments}}{\meta{底线注释}}选项提供\textbf{底线注释}内容。
%
%   代码盒子的外观也可以通过\tn{codeset}命令的逗号分隔
%   key-value列表进行设置。
% \end{function}
%
% \newpage
%   \begin{Verbatim}[frame=none,numbers=left,gobble=5]
%     \centering
%     \codefile{代码盒子标题}{test.c}
%   \end{Verbatim}
%
%   \begin{center}
%     \begin{minipage}{0.85\textwidth}
%       \codefile{代码盒子标题}{test.c}
%     \end{minipage}
%   \end{center}
%
% \subsection{\env{codeview}和\env{codeview*}环境}
%
% \begin{function}[added=2021-12-26,updated=2021-12-26]{codeview,codeview*}
%   \begin{syntax}
%     \tn{begin}\{codeview\}\oarg{外观选项}\Arg{盒子标题}
%     .....
%     \tn{end}\{codeview\}
%     \tn{begin}\{codeview*\}\oarg{外观选项}\Arg{盒子标题}
%     .....
%     \tn{end}\{codeview*\}
%   \end{syntax}
%   使用环境内容排版带编号代码盒子。
%   可以通过\Arg{盒子标题}为代码盒子设置标题。
%
%   \oarg{外观选项}中通过key-value方式设置代码盒子外观。
%
%   星号环境\texttt{codeview*}用于为代码盒子添加底线注释,
%   此时,需要在\oarg{外观选项}中用
%   \kvopt{\meta{comments}}{\meta{底线注释}}选项提供\textbf{底线注释}内容。
%
%   代码盒子的外观也可以通过\tn{codeset}命令的逗号分隔
%   key-value列表进行设置。
% \end{function}
%
%   \begin{Verbatim}[frame=none,numbers=left,gobble=5]
%     \centering
%     \begin{codeview}{代码标题}
%       #include <stdio.h>
%       #include <stdlib.h>
%
%       int main(void)
%       {
%           printf("Hello World!\n");
%
%           return 0;
%       }
%     \end{codeview}
%   \end{Verbatim}
%
%   \codeset{pretitle=代码}
%   \begin{center}
%     \begin{minipage}{0.85\textwidth}
%       \cvfile{代码标题}{test.c}
%     \end{minipage}
%   \end{center}
%
% \subsection{\tn{cvfile}和\tn{cvfile*}命令}
%
% \begin{function}[added=2021-12-26,updated=2021-12-26]{\cvfile,\cvfile*}
%   \begin{syntax}
%     \tn{cvfile}  \oarg{外观选项} \Arg{盒子标题} \Arg{代码文件}
%     \tn{cvfile*} \oarg{外观选项} \Arg{盒子标题} \Arg{代码文件}
%   \end{syntax}
%
%   用于排版带编号代码盒子,源代码来自于\Arg{代码文件},
%   可以通过\Arg{盒子标题}为代码盒子设置标题。
%
%   \oarg{外观选项}中通过key-value方式设置代码盒子外观。
%
%   星号命令\texttt{cvfile*}用于为代码盒子添加底线注释,
%   此时,需要在\oarg{外观选项}中用
%   \kvopt{\meta{comments}}{\meta{底线注释}}选项提供\textbf{底线注释}内容。
%
%   代码盒子的外观也可以通过\tn{codeset}命令的逗号分隔
%   key-value列表进行设置。
% \end{function}
%
%   \begin{Verbatim}[frame=none,numbers=left,gobble=5]
%     \centering
%     \cvfile*[comments=这是一个简单的C语言代码]{代码标题}{test.c}
%   \end{Verbatim}
%
%   \begin{center}
%     \begin{minipage}{0.85\textwidth}
%       \cvfile*[comments=这是一个简单的C语言代码]{代码标题}{test.c}
%     \end{minipage}
%   \end{center}
%
% \section{选项说明}
%
%   \pkg{codebox}宏包提供了一系列选项,以设置代码盒子的外观样式。
%   载入\pkg{codebox}宏包后,以下选项均可通过用户接口命令\tn{codeset}进行设置。
%   当然,这些选项也可以通过各个环境或命令的\oarg{外观选项}进行设置。
%
% \subsection{代码引擎}
%
% \begin{function}[added=2021-12-26,updated=2021-12-26]{minted}
%   \begin{syntax}
%     minted = <\TTF> \init{true}
%   \end{syntax}
%   \opt{minted} 用于设置代码排版引擎,如为\textbf{true}则使用\pkg{minted}宏包
%   排版代码盒子中的内容,如为\textbf{false}则使用\pkg{listings}宏包排版
%   代码盒子中的内容。默认值为\textbf{true}。
% \end{function}
%
% \subsection{代码语言}
%
% \begin{function}[added=2021-12-26,updated=2021-12-26]{lang}
%   \begin{syntax}
%     lang = \Arg{代码语言} \init{c}
%   \end{syntax}
%   \opt{lang} 用于设置代码语言,默认值为\textbf{c语言}。
% \end{function}
%
% \subsection{代码名称}
%
% \begin{function}[added=2021-12-26,updated=2021-12-26]{pretitle}
%   \begin{syntax}
%     pretitle = \Arg{代码名称} \init{Code}
%   \end{syntax}
%   \opt{pretitle} 用于代码编号前的代码名称前缀,默认值为\textbf{Code}。
% \end{function}
%
% \subsection{代码样式}
%
% \begin{function}[added=2021-12-26,updated=2021-12-26]{codestyle}
%   \begin{syntax}
%     codestyle = \Arg{代码样式} \init{codeblocks}
%   \end{syntax}
%   \opt{codestyle} 用于设置代码盒子中代码的样式,仅对\textbf{minted}
%   引擎有效,默认值为\textbf{codeblocks}。
% \end{function}
%
% \subsection{代码字号}
%
% \begin{function}[added=2021-12-26,updated=2021-12-26]{codesize}
%   \begin{syntax}
%     codesize = \Arg{字号命令} \init{\small}
%   \end{syntax}
%   \opt{codesize} 用于设置代码盒子中代码的字号,仅对\textbf{minted}
%   引擎有效,默认值为\textbf{\tn{small}}。
% \end{function}
%
% \subsection{底线注释内容}
%
% \begin{function}[added=2021-12-26,updated=2021-12-26]{comments}
%   \begin{syntax}
%     comments = \Arg{注释内容} \init{无}
%   \end{syntax}
%   \opt{commentf} 用于设置代码盒子底线注释的内容,
%   默认值为\textbf{无}。
% \end{function}
%
% \subsection{底线注释格式}
%
% \begin{function}[added=2021-12-26,updated=2021-12-26]{commentf}
%   \begin{syntax}
%     commentf = \Arg{格式命令组} \init{\small\sffamily}
%   \end{syntax}
%   \opt{commentf} 用于设置代码盒子底线注释的格式,
%   默认值为\textbf{\tn{small}\tn{sffamily}}。
% \end{function}
%
% \subsection{代码行距}
%
% \begin{function}[added=2021-12-26,updated=2021-12-26]{codestretch}
%   \begin{syntax}
%     codestretch = \Arg{浮点数} \init{1.0}
%   \end{syntax}
%   \opt{codestretch} 用于设置代码盒子内容(代码)的行距,仅对\textbf{minted}
%   引擎有效,取浮点数。默认值为\textbf{1.0}。
% \end{function}
%
% \subsection{行号间距}
%
% \begin{function}[added=2021-12-26,updated=2022-1-28]{linenumsep}
%   \begin{syntax}
%     linenumsep = \Arg{浮点数} \init{1.80}
%   \end{syntax}
%   \opt{linenumsep} 用于设置代码盒子行号与代码的间距,仅对\textbf{minted}
%   引擎有效,取浮点数,单位是mm。默认值为\textbf{3.0}。
% \end{function}
%
% \changes{v1.0.3}{2022/1/4}{添加label引用标签选项的使用说明}
%
% \subsection{引用标签}
%
% \begin{function}[added=2022-1-4,updated=2022-1-4]{label}
%   \begin{syntax}
%     label = \Arg{引用标签名称} \init{空}
%   \end{syntax}
%   \opt{label} 用于设置带编号代码盒子的引用标签,
%   该选项仅对\env{codeview}/\env{codeview*}环境
%   或\tn{cvfile}/\tn{cvfile*}命令有效。
%   默认值为\textbf{空}。
% \end{function}
%
% \changes{v1.0.2}{2021/12/28}{添加cvcounter使用说明}
%
% \section{计数器}
%
% \begin{function}[added=2021-12-28,updated=2021-12-28]{cvcounter}
%   \pkg{codebox}宏包提供了一个计数器\texttt{cvcounter},
%   当使用带编号环境\env{codeview}/\env{codeview*}和
%   命令\tn{cvfile}/\tn{cvfile*}排版代码时,可以用于对代码盒子进行计数。
%   默认情况下,如果存在\texttt{thechapter}计数器,则其父计数器设置为\texttt{chapter}
%   否则将按全文统一计数。
%
%   可以使用类似\tn{renewcommand\{\tn{thecvcounter}\}\{\tn{thechapter.\tn{arabic\{cvcounter\}}}\}}
%   的形式更改编号输出内容。
% \end{function}
%
% \newpage
% \section{排版样例}
%
% \pkg{codebox}宏包可用于需要语法高亮代码排版场合,
% 以避免使用代码截图。代码盒子可以带/不带底线注释。
%
% \subsection{Java代码}
% 可以使用\tn{codeset}命令设置代码语言。
%
%   \begin{Verbatim}[frame=none,numbers=left,gobble=5]
%     \centering
%     \codeset{lang=java}
%     \codefile{Java代码盒子}{hellojava.java}
%   \end{Verbatim}
%
%   \begin{center}
%     \begin{minipage}{0.85\textwidth}
%       \codeset{lang=java}
%       \codefile{Java代码盒子}{hellojava.java}
%     \end{minipage}
%   \end{center}
%
% \subsection{Python代码}
% 还可以使用环境或命令的\oarg{外观选项}设置代码语言。
% 当然,也可以通过使用\opt{label}选项为编号代码添加引用标签,
% 并通过\tn{ref}命令引用该标签,如代码\ref{code-test}所示。
%
%   \begin{Verbatim}[frame=none,numbers=left,gobble=5]
%     \centering
%     \cvfile[lang=python,label=code-test]{Python代码示例}{hellopy.py}
%   \end{Verbatim}
%
%   \begin{center}
% %     \begin{minipage}{0.85\textwidth}
%       \cvfile[lang=python,label=code-test]{Python代码示例}{hellopy.py}
%     \end{minipage}
%   \end{center}
%
% \subsection{listings引擎}
% 在选项中使用\kvopt{\meta{minted}}{\meta{false}}
% 以使用\pkg{listings}宏包排版代码。
%
%   \begin{Verbatim}[frame=none,numbers=left,gobble=5]
%     \centering
%     \cvfile[minted=false,lang=c]{C语言代码示例}{test.c}
%   \end{Verbatim}
%
%   \begin{center}
%     \begin{minipage}{0.85\textwidth}
%       \cvfile[minted=false,lang=c]{C语言代码示例}{test.c}
%     \end{minipage}
%   \end{center}
%
%
% \end{documentation}
%
% \StopEventually{}
%
% \begin{implementation}
%
% \section{代码实现}
%
% 该宏包使用 \LaTeX3 语法编写,依赖 \pkg{expl3} 环境,
% 并需调用 \pkg{l3packages}、\pkg{tcolorbox}、\pkg{minted}
% \pkg{listings}、\pkg{fontawesome5}等宏包。
%
% 按照 \LaTeX3 语法,代码中的空格、换行、回车与制表符会完全被忽略,
% 而下划线“|_|”和冒号“|:|”则可作为一般字母使用。
% 正常的空格可以使用“|~|”代替;至于 |~| 原来所表示的“带子”,
% 则要用 \LaTeXe{} 的原始命令 \tn{nobreakspace} 代替。
%
% 以下代码中有一些形如 \textcolor[HTML]{2E3191}{\textsf{<*package>}}
% 的标记,这是 \pkg{DocStrip} 中的“guard”,用来选择性地提取文件。
% “\textsf{*}”和“\textsf{/}”分别表示该部分的开始和结束。不含
% “\textsf{*}”和“\textsf{/}”的 guard 出现在行号右侧,它们用来确定
% 单独一行代码的归属。这些 guard 的颜色深浅不一,用以明确嵌套关系。
%
% 另有若干形如 \textcolor[HTML]{AD1457}{\textsf{<@@=codebox>}} 的 guard,
% 它们由 \pkg{l3docstrip} 定义,用来标识名字空间(模块)。
%
% \subsection{环境检测与准备}
%
%    \begin{macrocode}
%<*package>
%<@@=codebox>
%    \end{macrocode}
%
% 载入必要的\LaTeX3 宏包
%
%    \begin{macrocode}
\RequirePackage { xtemplate, l3keys2e, xparse }
%    \end{macrocode}
%
% \changes{v1.0.1}{2021/12/26}{删除原\LaTeX3 版本检测代码}
%
% \subsection{载入其它宏包}
%
% 载入\pkg{fontawesome5}、\pkg{tcolorbox}、\pkg{amssymb}、
% \pkg{xcolor}宏包
%
%    \begin{macrocode}
\RequirePackage {fontawesome5, tcolorbox, varwidth, xcolor, etoolbox}
%    \end{macrocode}
%
% 解决Expl3中\pkg{tcolorbox}和\pkg{tikz}附加库无法加载的
% 的补丁(需要\pkg{etoolbox}宏包支持)
%
% \url{https://tex.stackexchange.com/questions/57424/using-of-usetikzlibrary-in-an-expl3-package-fails}
%
%    \begin{macrocode}
\ExplSyntaxOff
\patchcmd
{\tcb@input@library@in}
  {%
   \input\tcbpkgprefix#1\relax%
  }
  {%
    \@pushfilename
    \input\tcbpkgprefix#1\relax%
    \@popfilename
  }
  {}{}

\patchcmd
{\pgfutil@InputIfFileExists}
  {\input #1}
  {%
    \@pushfilename
    \xdef\@currname{#1}%
    \input #1 %
    \@popfilename
  }
  {}{}
\ExplSyntaxOn
%    \end{macrocode}
%
% \changes{v1.0.1}{2021/12/26}{minted2.6发布,删除原补丁代码}
%
% 载入\pkg{tcolorbox}和\pkg{tikz}附加库
%
%    \begin{macrocode}
\tcbuselibrary{skins, xparse, breakable}
\tcbuselibrary{minted,listings}
\usetikzlibrary{shapes.geometric}
%    \end{macrocode}
%
% \subsection{定义颜色}
%
%    \begin{macrocode}
\definecolor{cvgrayc}{RGB}{247,247,247}
\definecolor{cvgray}{RGB}{220,220,220}
\definecolor{cvgrayb}{RGB}{153,153,153}
\definecolor{cvblue}{RGB}{223,238,255}
\definecolor{chengse}{RGB}{250,140,53}
%    \end{macrocode}
%
% \subsection{代码盒子样式设计}
%
% 代码盒子lang样式
%
%    \begin{macrocode}
\tcbset{%
  skin=enhanced,
  lang/.style={%
    breakable,%
    drop~shadow,%
    colframe=gray!75!black,%
    left=4.5mm,
    enhanced,%
    colframe=tcbcolback!60!black,%
    colback=white,%
    colbacktitle=tcbcolback!5!gray!10!white,%
    fonttitle=\bfseries,%
    coltitle=black,%
    attach~boxed~title~to~top~center={%
      yshift=-0.25mm-\tcboxedtitleheight/2,%
      yshifttext=2mm-\tcboxedtitleheight/2%
    },%
    attach~boxed~title~to~top~left={%
      xshift=1cm,%
      yshift*=1mm-\tcboxedtitleheight%
    },%
    varwidth~boxed~title*=-3cm,%
    boxed~title~style={%
      frame~code={%
        \path[fill=tcbcolback!30!black]([yshift=-1mm,xshift=-1mm]frame.north~west)%
          arc[start~angle=0,end~angle=180,radius=1mm]([yshift=-1mm,xshift=1mm]frame.north~east)%
          arc[start~angle=180,end~angle=0,radius=1mm];%
        \path[left~color=tcbcolback!60!black,right~color=tcbcolback!60!black,
          middle~color=tcbcolback!80!black]([xshift=-2mm]frame.north~west)%
          --([xshift=2mm]frame.north~east)[rounded~corners=1mm]%
          --([xshift=1mm,yshift=-1mm]frame.north~east)%
          --(frame.south~east)%
          --(frame.south~west)%
          --([xshift=-1mm,yshift=-1mm]frame.north~west)[sharp~corners]%
          --cycle;%
      },%
      interior~engine=empty%
    },%
    overlay={%
      \begin{tcbclipinterior}
        \fill[tcbcolback!80!black] (frame.south~west) rectangle
         ([xshift=5mm]frame.north~west);
      \end{tcbclipinterior}%
    },
  }%
}%
%    \end{macrocode}
%
% \changes{v1.0.4}{2022/1/28}{修订分页代码盒子无行号底色问题}
%
% 代码盒子cv样式
%
%    \begin{macrocode}
\tcbset{%
  skin=enhanced,
  cv/.style={
    boxrule=0.4mm,
    breakable,
    top=0mm,
    boxsep=1mm,
    drop~shadow,
    attach~boxed~title~to~top,
    colframe=blue!75!black,
    left=4.5mm,
    right=0mm,
    enhanced,
    colframe=tcbcolback!60!black,
    colback=white,
    colbacktitle=cvgray,
    fonttitle=\ttfamily,
    coltitle=black,
    underlay~boxed~title = {
      \begin{tcbclipinterior}
        \fill[cvblue] (frame.south~west) rectangle ([xshift=5.0mm,yshift=0mm]frame.north~west);
      \end{tcbclipinterior}
    },
    underlay~middle = {
      \begin{tcbclipinterior}
        \fill[cvblue] (frame.south~west) rectangle
          ([xshift=5.0mm,yshift=0mm]frame.north~west);
      \end{tcbclipinterior}
    },
    underlay~last = {
      \begin{tcbclipinterior}
        \fill[cvblue] (frame.south~west) rectangle
          ([xshift=5.0mm,yshift=0mm]frame.north~west);
      \end{tcbclipinterior}
    },
  }
}%
%    \end{macrocode}
%
% \subsection{内部变量声明}
%
% \begin{variable}{\l_@@_minted_bool,
% \l_@@_comment_bool,
% \l_@@_language_tl,
% \l_@@_code_style_tl,
% \l_@@_code_fontsize_tl,
% \l_@@_code_name_tl,
% \l_@@_comment_contents_tl,
% \l_@@_comment_format_tl,
% \l_@@_label_name_tl,
% \l_@@_baseline_stretch_fp,
% \l_@@_linenumber_sep_fp
%  }
% 定义变量。
%
% \changes{v1.0.3}{2022/1/4}{增加\tn{\_@@_label_name_tl}变量}
%
%    \begin{macrocode}
\bool_new:N   \l_@@_minted_bool
\bool_new:N   \l_@@_comment_bool

\tl_new:N     \l_@@_language_tl
\tl_new:N     \l_@@_code_style_tl
\tl_new:N     \l_@@_code_fontsize_tl
\tl_new:N     \l_@@_code_name_tl
\tl_new:N     \l_@@_comment_contents_tl
\tl_new:N     \l_@@_comment_format_tl
\tl_new:N     \l_@@_label_name_tl
\fp_new:N     \l_@@_baseline_stretch_fp
\fp_new:N     \l_@@_linenumber_sep_fp
%    \end{macrocode}
% \end{variable}
%
% \changes{v1.0.4}{2022/1/28}{修正linenumsep为默认值1.80以避免3位行号溢出边界}
%
% \subsection{选项处理}
%
% 定义 |codebox| 键值类。
%    \begin{macrocode}
\keys_define:nn { codebox }
  {
%    \end{macrocode}
%
% \begin{macro}{minted}
% 是否使用minted宏包排版代码,默认为true。
%    \begin{macrocode}
    minted .bool_set:N = \l_@@_minted_bool,
    minted .default:n = true,
    minted .initial:n = true,
%    \end{macrocode}
% \end{macro}
%
% \begin{macro}{lang}
% 设置排版代码语言,默认为c。
%    \begin{macrocode}
    lang .tl_set:N = \l_@@_language_tl,
    lang .initial:n = c,
%    \end{macrocode}
% \end{macro}
%
% \begin{macro}{pretitle}
% 代码名称前缀
%    \begin{macrocode}
    pretitle .tl_set:N = \l_@@_code_name_tl,
    pretitle .initial:n = Code,
%    \end{macrocode}
% \end{macro}
%
% \begin{macro}{codestyle}
% 代码样式
%    \begin{macrocode}
    codestyle .tl_set:N = \l_@@_code_style_tl,
    codestyle .initial:n = default,
%    \end{macrocode}
% \end{macro}
%
% \begin{macro}{codesize}
% 代码字号
%    \begin{macrocode}
    codesize .tl_set:N = \l_@@_code_fontsize_tl,
    codesize .initial:n = \small,
%    \end{macrocode}
% \end{macro}
%
% \begin{macro}{comments}
% 注释格式
%    \begin{macrocode}
    comments .tl_set:N = \l_@@_comment_contents_tl,
    comments .initial:n = {},
%    \end{macrocode}
% \end{macro}
%
% \begin{macro}{commentf}
% 注释格式
%    \begin{macrocode}
    commentf .tl_set:N = \l_@@_comment_format_tl,
    commentf .initial:n = \small\sffamily,
%    \end{macrocode}
% \end{macro}
%
% \begin{macro}{codestretch}
% 基线伸展系数
%    \begin{macrocode}
    codestretch .code:n = { \fp_set:Nn \l_@@_baseline_stretch_fp { #1 }
                         },
    codestretch .initial:n = 1.0,
%    \end{macrocode}
% \end{macro}
%
% \begin{macro}{linenumsep}
% 行号与文本之间的距离(单位:mm)
%    \begin{macrocode}
    linenumsep .fp_set:N  = \l_@@_linenumber_sep_fp,
    linenumsep .initial:n = 1.80,
%    \end{macrocode}
% \end{macro}
%
% \changes{v1.0.3}{2022/1/4}{增加引用标签label选项}
%
% \begin{macro}{label}
% 注释格式
%    \begin{macrocode}
    label .tl_set:N = \l_@@_label_name_tl,
    label .initial:n = {},
%    \end{macrocode}
% \end{macro}
%
% 处理未知选项。
%    \begin{macrocode}
    unknown .code:n = { \@@_error:n { unknown-option } }
  }
\msg_new:nnn { codebox } { unknown-option }
  { package~ option~ "\l_keys_key_tl"~ is~ unknown. }
%    \end{macrocode}
%
% \subsection{选项设置用户接口}
%
% \begin{macro}{\codeset}
%
% 选项设置用户接口。
%
%    \begin{macrocode}
\NewDocumentCommand \codeset { m }
  { \keys_set:nn { codebox } {#1} }
%    \end{macrocode}
% \end{macro}
%
% \subsection{内部函数}
%
% \changes{v1.0.2}{2022/1/2}{修改按章编号时cvcounter的编号方式}
%
% \begin{macro}{\@@_set_counter_parent:}
%
% 设置代码编号计数器的父计数器。
%
%    \begin{macrocode}
\cs_new:Npn \@@_set_counter_parent:
  {
    \ifdef{\thechapter}
      {
        \newcounter{cvcounter}[chapter]%
        \renewcommand{\thecvcounter}{\thechapter.\arabic{cvcounter}}
      }{
        \newcounter{cvcounter}%[section]%
      }
  }
%    \end{macrocode}
% \end{macro}
%
% \changes{v1.0.1}{2021/12/26}{调整minted style选项为参数实现}
% \changes{v1.0.2}{2021/12/28}{将minted/listings的参数设置独立为函数}
%
% \begin{macro}{\@@_minted_engine_tcbset:n #1}
% 设置代码盒子minted宏包参数。
%
%    \begin{macrocode}
\cs_new:Npn \@@_minted_engine_tcbset:n #1
  {
    \tcbset{
      listing~engine=minted,%
      minted~style=#1,
      minted~options={%
        autogobble,
        breaklines,%
        fontsize=\tl_use:N \l_@@_code_fontsize_tl,%
        baselinestretch=\fp_eval:n { \l_@@_baseline_stretch_fp },
        breaksymbolleft={},%
        linenos,%
        numbersep=\fp_eval:n { \l_@@_linenumber_sep_fp }~mm,
      },%
      minted~language=\tl_use:N \l_@@_language_tl%
    }
  }
\cs_generate_variant:Nn \@@_minted_engine_tcbset:n {V}
%    \end{macrocode}
% \end{macro}
%
% \begin{macro}{\@@_listings_engine_tcbset:}
% 设置代码盒子listings宏包参数。
%
%    \begin{macrocode}
\cs_new:Npn \@@_listings_engine_tcbset:
  {
    \lstset{% general command to set parameter(s)
      basicstyle=\ttfamily\tl_use:N \l_@@_code_fontsize_tl, % print whole listing small
      keywordstyle=\color{chengse}\bfseries,%
      identifierstyle=, % nothing happens
      breaklines=true,
      commentstyle=\itshape\color{gray},
      stringstyle=\ttfamily, % typewriter type for strings
      numbers=left,%
      numberstyle=\tl_use:N \l_@@_code_fontsize_tl\sffamily\color[rgb]{0.5,0.5,1.0},%
    }%
    \tcbset{
      overlay={%
        \begin{tcbclipinterior}
          \fill[tcbcolback!80!black] (frame.south~west) rectangle
                                     ([xshift=5mm]frame.north~west);
        \end{tcbclipinterior}%
      },%
      listing~options={
        language=\tl_use:N \l_@@_language_tl,
        numbersep=\fp_eval:n { \l_@@_linenumber_sep_fp }~mm,
        %linebackgroundcolor={\ifodd\value{lstnumber}\color{cvgray}\fi}
      }
    }%
  }
%    \end{macrocode}
% \end{macro}
%
% \begin{macro}{\@@_code_engine_tcbset:}
% 设置代码盒子引擎宏包参数。
%
%    \begin{macrocode}
\cs_new:Npn \@@_code_engine_tcbset:
  {
    \bool_if:NTF \l_@@_minted_bool
      {
        \@@_minted_engine_tcbset:V \l_@@_code_style_tl
      }{
        \@@_listings_engine_tcbset:
      }
  }
%    \end{macrocode}
% \end{macro}
%
% \begin{macro}{\@@_comment_tcbset:}
% 设置代码盒子底线注释。
%
%    \begin{macrocode}
\cs_new:Npn \@@_comment_tcbset:
  {
    \bool_if:NTF { \l_@@_comment_bool }
    {
      \tcbset{%
        listing~and~comment,%
        colbacklower=tcbcolback!5!yellow!10!white,%
        collower=tcbcolback!60!black,%
        comment={\tl_use:N \l_@@_comment_format_tl~
                 \tl_use:N \l_@@_comment_contents_tl},%
      }
    }{
      \tcbset{ listing~only,}
    }
  }
%    \end{macrocode}
% \end{macro}
%
% \changes{v1.0.1}{2021/12/27}{修订代码盒子前多一个逗号的问题}
% \changes{v1.0.4}{2022/1/28}{修改用户接口的{group_begin}到开始位置}
%
% \subsection{不计数代码盒子用户接口}
%
%  定义不带底线注释的代码盒子环境\env{codebox}
%
%    \begin{macrocode}
\NewDocumentEnvironment{ codebox }{O{} m }
  {
    \group_begin:
    \bool_set_false:N  \l_@@_comment_bool
    \keys_set:nn { codebox } { #1 }
    \@@_code_engine_tcbset:
    \@@_comment_tcbset:
    \tcbset{lang, title={#2} }%
    \tcblisting{}
  }{
    \endtcblisting
    \group_end:
  }
%    \end{macrocode}
%
%  定义带底线注释的代码盒子环境\env{codebox*}
%
%    \begin{macrocode}
\NewDocumentEnvironment{ codebox* }{O{} m }
  {
    \group_begin:
    \bool_set_false:N  \l_@@_comment_bool
    \keys_set:nn { codebox } { #1 }
    \@@_code_engine_tcbset:
    \@@_comment_tcbset:
    \tcbset{lang, title={#2} }%
    \tcblisting{}
  }{
    \endtcblisting
    \group_end:
  }
%    \end{macrocode}
%
%  定义带/不带底线注释的从文件读取内容的命令\tn{codefile}/\tn{codefile*}
%
%    \begin{macrocode}
\NewDocumentCommand \codefile { s O{} m m }
  {
    \group_begin:
    \IfBooleanTF{#1}
      {
        \bool_set_true:N  \l_@@_comment_bool
      }{
        \bool_set_false:N \l_@@_comment_bool
      }
    \keys_set:nn { codebox } { #2 }
    \@@_code_engine_tcbset:
    \@@_comment_tcbset:
    \tcbset{
      lang,
      title={#3},
      listing~file={#4},
    }%
    \tcbinputlisting{  }
    \group_end:
  }
%    \end{macrocode}
%
% \subsection{计数代码盒子用户接口}
%
%  设置代码盒子计算器
%
%    \begin{macrocode}
\@@_set_counter_parent:
%    \end{macrocode}
%
% \changes{v1.0.2}{2021/12/28}{为计数器两端添加\tn{nobreakspace}}
% \changes{v1.0.3}{2022/1/4}{为带编号代码盒子添加引用标签选项}
%
%  定义不带底线注释的代码盒子环境\env{codeview}
%
%    \begin{macrocode}
\DeclareTCBListing[use~counter=cvcounter]{codeviewaux}{m}
  {
    title={\tl_use:N \l_@@_code_name_tl\nobreakspace\thecvcounter{}:\nobreakspace~#1},
  }
\NewDocumentEnvironment{ codeview }{O{} m }
  {
    \group_begin:
    \bool_set_false:N  \l_@@_comment_bool
    \keys_set:nn { codebox } { #1 }
    \@@_code_engine_tcbset:
    \@@_comment_tcbset:
    \tl_if_empty:NF \l_@@_label_name_tl
    {
      \tcbset{
        label=\tl_use:N \l_@@_label_name_tl,
      }
    }
    \tcbset{
      cv,
      overlay~unbroken~and~first ={
        \begin{tcbclipinterior}
          \node[inner~sep=0pt,anchor=north~east,yshift=-3pt,xshift=-5pt,text=cvgrayb]
            at (frame.north~east){
              \ttfamily\faFile*\ \faCode\ \faCodeBranch\ \faCopy\ \faExternalLink*\
              \MakeUppercase{\tl_use:N \l_@@_language_tl}
            };
        \end{tcbclipinterior}
      }
    }%
    \codeviewaux{#2}
  }{
    \endcodeviewaux
    \group_end:
  }
%    \end{macrocode}
%
%
%  定义带底线注释的代码盒子环境\env{codeview*}
%
%    \begin{macrocode}
\NewDocumentEnvironment{ codeview* }{O{} m }
  {
    \group_begin:
    \bool_set_true:N  \l_@@_comment_bool
    \keys_set:nn { codebox } { #1 }
    \@@_code_engine_tcbset:
    \@@_comment_tcbset:
    \tl_if_empty:NF \l_@@_label_name_tl
    {
      \tcbset{
        label=\tl_use:N \l_@@_label_name_tl,
      }
    }
    \tcbset{
      cv,
      overlay~unbroken~and~first ={
        \begin{tcbclipinterior}
          \node[inner~sep=0pt,anchor=north~east,yshift=-3pt,xshift=-5pt,text=cvgrayb]
            at (frame.north~east){
              \ttfamily\faFile*\ \faCode\ \faCodeBranch\ \faCopy\ \faExternalLink*\ 
              \MakeUppercase{\tl_use:N \l_@@_language_tl}
            };
        \end{tcbclipinterior}
      }
    }%
    \codeviewaux{#2}
  }{
    \endcodeviewaux
    \group_end:
  }
%    \end{macrocode}
%
%  定义带/不带底线注释的从文件读取内容的命令\tn{cvfile}/\tn{cvfile*}
%
%    \begin{macrocode}
\DeclareTCBInputListing[use~counter=cvcounter]{\langfileaux}{m}
  {
    title={\tl_use:N \l_@@_code_name_tl\nobreakspace\thecvcounter{}:\nobreakspace~#1},
  }
\NewDocumentCommand \cvfile { s O{} m m }
  {
    \group_begin:
    \IfBooleanTF{#1}
      {
        \bool_set_true:N  \l_@@_comment_bool
      }{
        \bool_set_false:N \l_@@_comment_bool
      }
    \keys_set:nn { codebox } { #2 }
    \@@_code_engine_tcbset:
    \@@_comment_tcbset:
    \tl_if_empty:NF \l_@@_label_name_tl
    {
      \tcbset{
        label=\tl_use:N \l_@@_label_name_tl,
      }
    }
    \tcbset{
      cv,
      listing~file={#4},
      overlay~unbroken~and~first ={
        \begin{tcbclipinterior}
          \node[inner~sep=0pt,anchor=north~east,yshift=-3pt,xshift=-5pt,text=cvgrayb]
            at (frame.north~east){
              \ttfamily\faFile*\ \faCode\ \faCodeBranch\ \faCopy\ \faExternalLink*\ 
              \MakeUppercase{\tl_use:N \l_@@_language_tl}
            };
        \end{tcbclipinterior}
      }
    }%
    \langfileaux{#3}
    \group_end:
  }
%    \end{macrocode}
%
%    \begin{macrocode}
%</package>
%    \end{macrocode}
%
% \end{implementation}
%
% \Finale
%
\endinput
